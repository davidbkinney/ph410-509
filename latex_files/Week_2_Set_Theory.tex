\documentclass[11pt]{article}

% Packages for math and proof writing
\usepackage{amsmath, amssymb, amsthm}
\usepackage{bbm}

% Theorem and proof environments
\newtheorem{theorem}{Theorem}
\newtheorem{lemma}[theorem]{Lemma}
\newtheorem{proposition}[theorem]{Proposition}
\newtheorem{corollary}[theorem]{Corollary}

% Definitions, remarks, etc.
\theoremstyle{definition}
\newtheorem{definition}[theorem]{Definition}
\newtheorem{example}[theorem]{Example}

\theoremstyle{remark}
\newtheorem{remark}[theorem]{Remark}

% Page formatting
\usepackage[margin=1in]{geometry}

\title{Week 2:\ Set Theory}
\author{David Kinney}
\date{September 4th, 2025}

\begin{document}

\maketitle

\section{The Basics}
Consider the universe $U$ of all sets. For now, we won't say what a set is, though we will eventually provide an axiomatic definition of a set. Following convention, we will in different contexts use lower-case letters (e.g., $x$, $y$, $z$) or upper-case letters (e.g., $X$, $Y$, $Z$) to represent sets. In what follows, we will discuss numbers, tuples, etc. For now, just trust that all of these objects are, perhaps covertly, sets.

\subsection{The `Is an Element of' Relation}
The first important symbol that we will introduce is the symbol $\in$, which represents the inclusion of one set in another. So, if a set $x$ is in the set $X$, then we will write $x\in X$ and say that $x$ is an \textbf{element} of $X$. By contrast, for any set $z$ and set $X$, $z\not\in X$ means that $z$ is \textit{not} in $X$.

\subsection{Set-Builder Notation}
One of often defines sets in philosophy using natural language (e.g., `let $P$ be the set of all permissible actions'). But sometimes, it is helpful to use \textbf{set-builder notation} to define sets. Here are the steps to defining a set using set-builder notation:
\begin{enumerate}
    \item Write an open bracket $\{$ to signal that the definition of the set is beginning.

    \item Write a statement of the types of sets that could belong in the set. For example, if know your set contains only real numbers, you might use $\mathbbm{R}$ to represent the set of real numbers and write $x\in\mathbbm{R}$ to signal that your set contains only real numbers.

    \item Write a $|$ or $:$ symbol to signal that your set will now be defined. One sees both symbols used in the literature. Either symbol can be translated into English as `such that.' 

    \item Write a statement describing exactly what kinds of objects do or do not belong in your set. This can be done in either natural language or symbols. For example, if one wanted to define a set containing only permissible actions, one might just write here `$x$ is permissible.' If one wanted to define a set containing only real numbers less than or equal to $2$, one might write `$x\leq 2$.'

    \item Write a close bracket $\}$ to signal the end of the sentence.
\end{enumerate}
Here are some examples. If we had a set of actions $A$ and wanted to define the subset that are permissible, we could write:
$\{a\in A : a \ \text{is permissible}\}.$
We would translate this into English as `the set of actions $a$ in $A$ such that $a$ is permissible.' If we wanted to define the set of real numbers less than or equal to $2$, we would write $\{x\in\mathbbm{R}:x\leq 2\}$.\par 

\section{Basic Concepts}

\subsection{The Empty Set}
The empty set is the set that contains no elements. Though it may seem trivial, the empty set plays a crucial role in many applications of set theory. The empty set is written $\emptyset$.

\subsection{Identity}
For two sets $X$ and $Y$, if every element of $X$ is an element of $Y$ and every element of $Y$ is an element of $X$, then we say that $X$ and $Y$ are identical. We represent this in symbols as $X=Y$.

\subsection{Subset}
Let $X$ and $Y$ be two sets. $X$ is a \textbf{subset} of $Y$ if every element of $X$ is also an element of $Y$. We represent this relation as $X\subseteq Y$. Note that if $X=Y$, then $X\subseteq Y$ and $Y\subseteq X$. 

\subsection{Proper Subset}
Let $X$ be a subset of $Y$. If, in addition, $X\neq Y$, then $X$ is a \textbf{proper subset} of $Y$. This means that any element of $X$ is an element of $Y$, but there are elements of $Y$ that are not elements of $X$. 

\subsection{Union}
Let $X$ and $Y$ be two sets. Their \textbf{union} is a third set, written $X\cup Y$, that contains all and only the elements of either $X$ or $Y$. More precisely, we can define $X\cup Y$ using set-builder notation as follows:
$$X\cup Y = \{z\in U: z\in X \ \text{or} \ z\in Y\}.$$

\subsection{Intersection}
Let $X$ and $Y$ be two sets. Their \textbf{intersection} is a third set, written $X\cap Y$, that contains all and only the elements of both $X$ and $Y$. More precisely, we can define $X\cap Y$ using set-builder notation as follows:
$$X\cap Y = \{z\in U: z\in X \ \text{and} \ z\in Y\}.$$

\subsection{Complement}
Let $Y$ be a set and consider $X\subseteq Y$. The \textbf{complement} of $X$ in $Y$, written $X^{c}$, is the set of all elements of $Y$ that are not in $X$. In set-builder notion:
$$X^{c}=\{z\in Y:z\not\in X\}.$$
Even though it is not clear from the notation, the complement of any set $X$ is always defined in relation to some larger set of which $X$ is a subset. Where this is not obvious from context, it should be stated explicitly. 

\subsection{Set Minus}
For any sets $X$ and $Y$, we write $X\setminus Y$ to represent the elements of $X$ that are \textit{not} in $Y$. In other words, $X\setminus Y$ is the result of \textbf{subtracting} the elements of $Y$ from the set $X$. In set-builder notion:
$$X\setminus Y = \{z:z\in X \ \text{and} \ z\not\in Y\}.$$

\subsection{Power Set}
For any set $X$, its \textbf{power set} $\mathcal{P}(X)$ is the set of all subsets of $X$. For example, if $X=\{x,y,z\}$, then $\mathcal{P}(X)=\{\emptyset,\{x\},\{y\},\{z\},\{x,y\},\{y,z\},\{x,z\},\{x,y,z\}\}$. Importantly, elements of $X$ are not necessarily members of its power set $\mathcal{P}(X)$. In this case, for example, $x\not\in\mathcal{P}(X)$. However, it \textit{is} the case that $\{x\}\in\mathcal{P}(X)$. This example shows that we must be careful to distinguish a set from the set containing only that set.

\subsection{Tuples}
A \textbf{tuple} is a collection of more than one set whose order matters. Tuples are distinguished from ordinary sets via the use of parenthesis rather than brackets. Any example of a tuple is $(a,b)$, where $a$ and $b$ are objects. As stated above, the order of a set matters, and so $(a,b)$ is not the same tuple as $(b,a)$, even though they have the same number of objects. This is the primary characteristic of tuples that distinguishes them from ordinary sets. Tuples containing two objects are called \textbf{pairs}, tuples containing three objects are called \textbf{triples}. For all $n>3$, a tuple containing $n$ objects is called an $n$-tuple.\par 

All tuples are part of a universe $U$ of sets. This may seem counterintuitive given that the order of objects in a tuple matters. Here is how we can nevertheless define a tuple as a set, using a technique developed by Kazimierz Kuratowski (1896-1980). Let any pair $(a,b)$ effectively be a shorthand for a set $\{\{a\},\{a,b\}\}$. That is, a pair is a set with two elements:\ 1) a set containing only the first element on the pair, and 2) a set containing both elements of the pair. Thus, $(a,b)$ is not the same set as $(b,a)$, since $(b,a)$ is now effectively shorthand for the set $\{\{b\},\{a,b\}\}$. Next, let any triple $(a,b,c)$ be a shorthand for a pair whose first element is the pair containing the first two elements of $(a,b,c)$ and whose second element is the third element of the triple; we write this pair $((a,b),c)$. Thus, the triple can now be written as a set:
$$(a,b,c)=((a,b),c)=\{\{(a,b)\},\{(a,b),c\}\}=\{\{\{\{a\},\{a,b\}\}\},\{\{\{a\},\{a,b\}\},c\}\}.$$
A 4-tuple $(a,b,c,d)$ can be written as a pair whose first element is the triple $(a,b,c)$ and whose second element is $d$; we write this pair $((a,b,c),d)$. Using the earlier rule for converting triples to pairs, we can re-write $(((a,b),c),d)$, at which point we can replace each pair with its set-theoretic definition from the outside in, as follows:
$$(((a,b),c),d)=\{\{((a,b),c)\},\{((a,b),c),d\}\}=$$
$$\{\{\{\{\{\{a\},\{a,b\}\}\},\{\{\{a\},\{a,b\}\},c\}\}\},\{\{\{\{\{a\},\{a,b\}\}\},\{\{\{a\},\{a,b\}\},c\}\},d\}\}.$$
We can repeat this process for any $n$-tuple to turn it into a set.\par 

\subsection{Relations}
A \textbf{relation} is a set of tuples, usually such that each tuple has the same number of objects in it. A collection of pairs is called a two-place relation, a collection of triples is called a three-place relation, and a collection of $n$-tuples is called an $n$-place relation. To see the connection to the typical philosophical understanding of a relation, consider the greater-than relation among real numbers. This relation can be defined in set-builder notation as $G = \{(a,b)\in\mathbbm{R}^{2} : a>b\}$, where $\mathbbm{R}^{2}$ is the set of all pairs of real numbers. We then can say of any two real numbers $x$ and $y$ that $x$ is greater than $y$ if and only if $(x,y)\in G$. 

\subsection{Products}
Let $X$ and $Y$ be sets. Their \textbf{product} $X\times Y$ is the set of all pairs whose first object is an element of $X$ and whose second object is an element of $Y$. In set-builder notation:
$$X\times Y = \{(x,y)\in U: x\in X \ \text{and} \ y\in Y\}.$$
The order in which a product is constructed matters. For example, the product 
$$Y\times X = \{(y,x)\in U:x\in X \ \text{and} \ y\in Y\}$$ 
is not generally the same set as $X\times Y$. Products can be generated from $n$ sets to form sets of $n$-tuples. Some examples: 
$$X\times Y \times Z = \{(x,y,z)\in U:x\in X \ \text{and} \ y\in Y \ \text{and} \ z\in Z\},$$
$$X\times Y \times Z \times W = \{(x,y,z,w)\in U:x\in X \ \text{and} \ y\in Y \ \text{and} \ z\in Z \ \text{and} \ w\in W\}.$$
Thus, a product is a special type of relation.\par

One can take the product of a set with itself $n$ times. For a set $X$, such a product is written $X^{n}$. For example:
$$X^{2} = X\times X = \{(x,x^{\prime})\in U:x\in X \ \text{and} \ x^{\prime}\in X\},$$
$$X^{3} = X\times X\times X = \{(x,x^{\prime},x^{\dagger})\in U:x\in X \ \text{and} \ x^{\prime}\in X \ \text{and} \ x^{\dagger}\in X\}.$$
One often sees this in applications of set theory in mathematics or mathematical philosophy, where, as we saw above, $\mathbbm{R}^{2}$ is the set of all pairs of real numbers.

\subsection{Functions}
A \textbf{function} is a way of assigning each objects in one set to an object in another set. When introducing a function, we typically:
\begin{enumerate}
    \item Write the letter representing the function (e.g., $f$).

    \item Write $:$.

    \item Write the set such that every element in that set is assigned to an element in another set (this first set is called the \textbf{domain} of the function).

    \item Write $\rightarrow$. 

    \item Write the set containing elements to which elements in the domain are assigned (this second set is called the \textbf{range} of the function).
\end{enumerate}
For example, a function $f:X\rightarrow Y$ is a function such that every element of $X$ is assigned to (the term `mapped to' is also sometimes used) an element of $Y$. The set $X$ is the domain of $f$, and the set $Y$ is the range of $f$. As an example, addition on the real numbers could be introduced as a function $a:\mathbbm{R}^{2}\rightarrow \mathbbm{R}$. That is, it assigns to every pair of real numbers a single real number (i.e., their sum). For any $x$ in the domain of a function $f$, we write $f(x)$ for the element of the range that $f$ assigns to $x$.\par

Sometimes, we wish to be more explicit about exactly how a function $f$ maps elements of its domain to its range. In such cases, we can define the function using the $\mapsto$ symbol. As an example, let $s:\mathbbm{R}\rightarrow\mathbbm{R}$ be the square function defined on the real numbers. We can write this function as $s:x\mapsto x^{2}$. This makes it explicit that $s$ assigns each real number to its square.\par 

For any function $f:X\rightarrow Y$, we can define its \textbf{inverse} as a function $f^{-1}:Y\rightarrow\mathcal{P}(X)$. That is, it maps elements of $Y$ to subsets of $X$. The explicit form of this mapping is: $$f^{-1}:y\mapsto\{x\in X: f(x)=y\}.$$ In other words, $f^{-1}(y)$ is the set of all $x$ such that $f(x)=y$.\par 

For a given function $f:X\rightarrow Y$, it need not be the case that every element $y$ of the range $Y$ be such that the inverse $f^{-1}(y)$ is non-empty. Consider again the function $s:\mathbbm{R}\rightarrow\mathbbm{R}$, with the explicit mapping $s:x\mapsto x^{2}$. Since squares are always positive, for any $r\in\mathbbm{R}$ such that $r$ is negative, $s^{-1}(r)=\emptyset$. The subset of a functions range containing only those elements of the range with non-empty inverse is called the \textbf{codomain}. More precisely, for any function $f:X\rightarrow Y$, its codomain $\textsf{Cod}(f)$ is
$\{y\in Y:f^{-1}(y)\neq\emptyset\}$.\par 


A function $f:X\rightarrow Y$ is called \textbf{surjective} if and only if $\textsf{Cod}(f)=Y$. That is, a function is surjective if and only if every element of the range has at least one element of the domain assigned to it. A function $f:X\rightarrow Y$ is called \textbf{injective} if and only if, for every $y\in Y$, $f^{-1}(y)$ is either the empty set or contains only one element. That is, a function is injective if every element of the range is such that, at most, at least one element of the domain is mapped to it. A function is called \textbf{bijective} if and only if it is both injective and surjective.\par 

A function $f:X\rightarrow Y$ can be re-written as the set $\{(x,y)\in X\times Y : y=f(x)\}$. Thus, functions too are sets.\par 

\section{Russell's Paradox}
Set theory was first used in Georg Cantor's (1845-1918) groundbreaking work on transfinite cardinalities, which we'll discuss briefly below. However, Cantor's set theory was \textit{naive} in the sense that it assumed that for any predicate of natural language, there was a set $X$ in the universe of all sets such that all and only the elements $x\in X$ satisfy the predicate. This was called the \textbf{unrestricted comprehension principle}. In 1901, Bertrand Russell (1872-1970) discovered a paradox that undermines the unrestricted comprehension principle. Consider the predicate `is a set that does not contain itself.' We can ostensibly use set-builder notation to define the \textbf{Russell set} $R$ as follows:
$$R=\{x\in U:x\not\in x\}.$$
This is the set of all sets that do not contain themselves. Now consider the question:\ is $R$ an element of itself? If it is not, then it follows that $R\not\in R$. So, we might conclude that $R\not\in R$. But, of course, if $R\not\in R$ then $R$ \textit{is} an element of $R$. So, we seemed to have derived that $R\in R$ if and only if $R\not\in R$. This famous paradox is taken to demonstrate the inherent inconsistency of the unrestricted comprehension principle.\par 


Russell's paradox, which may have been independently discovered by Ernst Zermelo (1871-1953), revealed that set theory would need firmer axiomatic foundations than the unrestricted comprehension principle. Zermelo began working on said axioms, and his work was taken up by Abraham Fraenkl (1891-1965). Their work has culminated in the now-standard \textbf{ZFC} (Zermelo-Fraenkl with Choice) axioms, which we will discuss in detail in the next section.\par 


\section{The ZFC Axioms}
The ZFC axioms are meant to define the set-theoretic universe. That is, the universe $U$ of sets is just the collection of mathematical objects that satisfies these ten axioms. It turns out that these axioms are sufficient to avoid Russell's paradox. In what follows, we explain all ten axioms. The axioms are written in a first-order logic defined over the domain of sets. In addition to the standard symbols of first-order logic, the axioms introduce a few others that we will explain as we go.

\subsection{Extensionality}
This axiom just states that when two sets have the same elements, they are the same set. In first-order logic: 
$$\forall x\forall y \left[\forall z (z\in x \iff z\in y)\Rightarrow x=y\right].$$

\subsection{Null Set}
This axiom just states that the empty set is part of the set-theoretic universe. In first-order logic:
$$\exists x \neg\exists y (y\in x).$$

\subsection{Pairing}
This axiom just states that for any two sets, we can construct a third set that contains only those two sets as elements. In first-order logic:
$$\forall x \forall y \exists z \forall w \left(w\in z \iff (w=x \vee w=y)\right).$$
The terminology here is confusing. The pairing axioms tells us we can define a set containing any two sets. But it does \textit{not} say that we can define a pair (in the sense of a two-element tuple) containing any two elements. We will have to live with this confusion.

\subsection{Power Set}
This axiom just states that every set has a power set. In first-order logic:
$$\forall x \exists y \forall z \left[z\in y \iff \forall w (w\in z \Rightarrow w\in x)\right].$$
Note that the formula $\forall w (w\in z \to w\in x)$ just states that $z\subseteq x$ (for all $w$, if $w$ is in $z$, then $w$ is in $x$). So, the power set axiom just says that for all sets $z$, there exists a set $y$ such that, for all sets $z$, $z$ is an element of $y$ if and only if it is a subset of $x$. This is just the power set. One can prove that every set has a unique power set (and you will do this below). 


\subsection{Union}
This axiom just states that for any set, we can define a third set that is the union of all sets in that set. In first-order logic:
$$\forall x \exists y \forall z \left[z\in y \iff \exists w (w\in x \wedge z\in w)\right].$$
Since the union of any set of sets is provably unique, we can, as we have done above, write $x\cup y$ for the union of two sets.


\subsection{Infinity}
This axiom provides a procedure for constructing a set with infinitely many elements. In a first-order logic equipped with the union relation $\cup$, the axiom is written as follows:
$$\exists x \left[\emptyset \in x \wedge \forall y (y\in x \Rightarrow y\cup \{y\}\in x)\right].$$
To see what this axiom enables us to do, consider the set $x$ that it says exists. The axiom also says that $\emptyset\in x$. The axiom also says that $\emptyset \cup \{\emptyset\}$, which we can write $\{\emptyset,\{\emptyset\}\}$, is in $x$. But then it also says that $\{\emptyset,\{\emptyset\}\}\cup\{\{\emptyset,\{\emptyset\}\}\}$, which we can write $\{\{\emptyset,\{\emptyset\}\},\{\{\emptyset,\{\emptyset\}\}\}$, is in $x$. It should be clear that we could keep doing this forever, always defining new elements of $x$. So, there is no finite upper bound to the number of elements in $x$. 


\subsection{Separation}
This axiom is slightly more complicated. As a warm-up, let $\varphi(x)$ be a formula in first-order logic defined over the domain of sets with the single free variable $x$. Now suppose we use set-builder notation to define a set $\{x\in z : \varphi(x)\}$. This is just the subset of a set $z$ whose elements satisfy the formula $\varphi(x)$. As an example, $\varphi(x)$ could be the formula `$x$ is a multiple of 2.' One could then define the subset of the real numbers that are even integers using the following notation:\ $\{q\in\mathbbm{R}:\varphi(q)\}$.\par 

The separation axiom provides a general statement of the kinds of conditions under which we can construct such a set. Let $\varphi(x,u_{1},\dots,u_{n})$ be a formula of first-order logic, defined over the domain of all sets, such that $x$ is a free variables, each of $u_{1},\dots,u_{n}$ may be a free variable, and no other variables are free in $\varphi$. The separation axiom says that for any set $z$, we can define a subset $r\subseteq z$ containing all and only those sets in $z$ that can be plugged in for $x$ to satisfy $\varphi(x,u_{1},\dots,u_{n})$. In first order logic:
$$\forall z\forall u_{1}\dots\forall u_{n}\exists y \forall r \left[r\in y \iff \left(r\in z \wedge \varphi(r,u_{1},\dots,u_{n}) \right) \right].$$
In other words, for any formula $\varphi(x,u_{1},\dots,u_{n})$, any set $z$, and any sets $u_{1},\dots,u_{n}$, we can define a set $\{r\in z : \varphi(r,u_{1},\dots,u_{1})\}$. In other words, for any set $z$, we can \textit{separate} the elements of $z$ that satisfy some formula from the rest of $z$. 

\subsection{Replacement}
This axiom uses similar formalism to separation. Let $\phi(x,y,u_{1},\dots,u_{n})$ be a be a formula of first-order logic, defined over the domain of all sets, such that $x$ and $y$ are free variables, each of $u_{1},\dots,u_{n}$ may be a free variable, and no other variables are free in $\phi$. The axiom of replacement states that if there is a exactly one $y$ such that $\phi(x,y,u_{1},\dots,u_{n})$ is satisfied for a given $x$ and $u_{1},\dots,u_{n}$, then for any sets $r$ and $z$, there is a set $v$ consisting of all and only the sets $r\in z$ such that $\phi(s,r,u_{1},\dots,u_{n})$ is true for some $u_{1},\dots,u_{n}$. In first-order logic:
$$\forall x\forall u_{1}\dots\forall u_{n}\left[\left(\exists!y \ \phi(x,y,u_{1},\dots,u_{n})\right)\Rightarrow \left(\forall z \exists v \forall r \ r\in v \iff \exists s (s\in z \wedge \varphi(s,r,u_{1},\dots,u_{n}))\right) \right].$$
There is a close connection between the axiom of replacement and functions. One can think of a formula $\phi(x,y,u_{1},\dots,u_{n})$ that is satisfied by exactly one $y$ as a function $f_{\phi}$ that maps a tuple $(x,u_{1},\dots,u_{n})$ to the set $r$ that such that $\phi(x,r,u_{1},\dots,u_{n})$ is true. The axiom of replacement just says that if such a function $f_{\phi}$ exists, then for any tuple $u_{1},\dots,u_{n}$ and any $r$ there exists a set $\{s:f_{\phi}(s,u_{1},\dots,u_{n})=r\}$. In other words, the image $f^{-1}_{\phi}(r)$ exists.\par 

The axiom of replacement is so named because it allows you define functions that \textit{replace} one set with another. This is important in many mathematical applications of set theory.\par


\subsection{Regularity}
This axiom ensures that any non-empty set $x$ has a minimal element $y$, i.e., a set that contains no other elements of $x$. In first-order logic:
$$\forall x \left[x\neq \emptyset \Rightarrow \exists y \left(y\in x \wedge \forall z (z\in x \Rightarrow z\not\in y)\right)\right].$$



\subsection{Choice}
This final axiom, which has historically been the most controversial, is the axiom of choice. It states that for any set $x$ such that any two non-identical, non-empty sets $y\in x$ and $y^{\prime}\in x$ have no common elements, there is a set that contains exactly one element of each $y\in x$. In other words, if $x$ is a set of mutually exclusive sets, we can form a new set by \textit{choosing} exactly one element from each set in $x$. In first order logic:
\small
$$\forall x \left[\forall y\forall y^{\prime}(y\in x \wedge y^{\prime} \in x \wedge y\neq\emptyset \wedge y^{\prime} \neq \emptyset \wedge y\neq y^{\prime}
) \Rightarrow y\cap y^{\prime} = \emptyset \right]\Rightarrow \exists c\forall w\left[w\in x\Rightarrow(\exists!z(z\in w \wedge z\in c) \right].$$
\normalsize
Though it is beyond the scope of this course, it turns out the axiom of choice is needed to prove \textit{a lot} of important theorems in mathematics. However, it is controversial because it often allows us to prove the existence of mathematical objects without constructing a specific example of them.


\subsection{What the Axioms Do}
If one adopts ZFC as the axioms of set theory, then one is committed to the idea that if a putative set violates any of these axioms, then it is not a set, and if it does not violate any of these axioms, then it is a set. You will prove in the exercises that this allows us to rule out the paradox brought on by the Russell set.


\section{Numbers as Sets}
Above, we noted that in set theory, numbers are sets. Let us now make this point explicit. This basic introduction to numbers as sets should serve as an illustration of the way in which set theory can provide firm foundations for mathematics.

\subsection{Ordinals}
In mathematics and ordinary life, we can use numbers to \textbf{order} or \textbf{rank} the elements in a set. That is, we can say which element is first, which one is second, which one is third, etc. \textbf{Ordinal numbers} are numbers used to rank elements in a set. For practical purposes, one can \textit{start} by thinking of ordinal numbers as the non-negative counting numbers 0, 1, $\dots$, though we will soon introduce infinite ordinals.\par

John von Neumann (1903 - 1957) introduced a way of writing ordinal numbers as sets. We begin by letting 0 be the empty set (i.e., let $0=\emptyset$). From there, we define a \textbf{successor} function on sets such that $s:x\mapsto x\cup\{x\}$. This gives a set-theoretic definition of every finite ordinal. Recall that $0=\emptyset$. This means that the successor of 0, is $s(0)=(\emptyset)=\emptyset\cup\{\emptyset\}=\{\emptyset\}$. We call the successor of 0, which is the ordinal 1, is $\{\emptyset\}$. To define 2, we obtain $s(1)=s(\{\emptyset\})=\{\emptyset\}\cup\{\{\emptyset\}\}=\{\emptyset,\{\emptyset\}\}$. Clearly, we can iterate this process arbitrarily many times to obtain any finite ordinal. Notice that $1=\{0\}$ and $2=\{0,1\}$. You will prove in the exercises that for any finite ordinal $n$, $n=\{0,1,\dots,n-1\}$. This establishes an important connection between ordinal ranking and the relation $\in$. For any ordinal $n$, if $m<n$, then $m\in n$. In this case, we say that $m$ is less than $n$ and $n$ is greater than $m$. The finite ordinals are just the \textbf{natural numbers} $\mathbbm{N}$.\par


The successor relation allows us to define an \textbf{addition} function on von Neumann ordinals. First, for any set $N$ of von Neumann ordinals, let the \textbf{supremum} of $N$, written $\textsf{sup}(N)$, be the union of all sets in $N$. To illustrate, let $N=\{0,1,2\}=3$. This can be written as the following set:
$$N=\{0,1,2\}=\{\emptyset,\{\emptyset\},\{\emptyset,\{\emptyset\}\}\}.$$
The \textbf{supremum} of $N$ is:
$$\textsf{sup}(N)=\{\emptyset,\{\emptyset\}\}=2.$$
Now we can define an addition function $g$ on a pairs $(a,b)$ of von Neumann ordinals (recall that $s$ is the successor function).:
\begin{equation}
    g:(a,b)\mapsto a+b = \begin{cases}
        a & \text{if} \ b=0\\
        s(a+c) & \text{if} \ b=c+1\\
        \textsf{sup}(\{a+c:c<b\}) & \text{otherwise}\\
    \end{cases}
\end{equation}
To illustrate in a finite case, consider $g(2,3)$. Since $3=2+1$, we know that $g(2,3)=s(g(2,2))$. So, we'll need to calculate $g(2,2)$. Since $2=1+1$, $g(2,2)=s(g(2,1))$. So, we'll need to calculate $g(2,1)$. Since $1=0+1$, $g(2,1)=s(g(2,0))$. So, we'll need to calculate $g(2,0)$, which is $2$. Putting this all together, we get:
$$g(2,3)=s(g(2,2))=s(s(g(2,1)))=s(s(s(g(2,0))))=s(s(s(2)))=5.$$
There are infinite ordinals as well as finite ordinals. Let $\omega$ be the set of all finite ordinals. This is the least infinite ordinal. All finite ordinals are elements of $\omega$, but no infinite ordinal is an element of $\omega$. We can construct ordinals greater than $\omega$ via addition. For example, the ordinal $\omega + 1$ is the value of $g(\omega,1)$ defined above. You will prove in the exercises that $\omega\in(\omega+1)$ but $(\omega + 1)\not\in\omega$. It should be clear at this point that there are infinitely many infinite ordinals. Note that addition on the ordinals is \textbf{commutative}:\ $a+b=b+a$.


\subsection{Integers}
In mathematics, the \textbf{integers} $\mathbbm{Z}$ are the negative and positive whole numbers, along with zero. In set theory, integers are defined as sets of pairs of natural numbers that all share a common property. Specifically, for any natural number $n$, the corresponding positive integer $+n$ can be defined as a set of pairs:
$$+n=\{(a,b)\in\mathbbm{N}^{2}:n+b=a\}.$$ For example, the positive integer $+3$ is the set $$\{(a,b)\in\mathbbm{N}^{2}:3+b=a\}.$$ By contrast, the negative integer $-n$ is the set $$-n=\{(a,b)\in\mathbbm{N}^{2}:a+n=b\}.$$ For example, the negative integer $-3$ is the set: $$\{(a,b)\in\mathbbm{N}^{2}:a+3=b\}.$$
To give a more unified definition of the integers, take any integer $i$, positive or negative, and pick one element $(a,b)$ of $i$. The integer $i$ can be written as follows:
$$i=[(a,b)]=\{(c,d):a+d = b + c\}.$$
One can think of this as defining integers in terms of pairs of natural numbers $(a,b)$ whose \textit{difference} $a-b$ is that integer. So $3$ is the set of pairs $\{(3,0),(4,1),\dots\}$ and $-3$ is the set of pairs $\{(0,3),(1,4),\dots\}$.

We can extend the addition function to pairs of integers. Let $i$ and $j$ be integers. Their sum $i+j$ is the set
$$\{(a,b)\in\mathbbm{N}^{2}:\exists(c,d)\exists(r,s) \left[(c,d)\in i \wedge (r,s)\in j\wedge g(c,r)=a\wedge g(d,s)=b\right]\},$$ 
where $g$ is the addition function on the ordinals. Addition on the integers is also commutative. One integer $i$ is greater than another integer $i^{\prime}$ if and only if: 1) $i$ is positive and $i^{\prime}$ is negative, 2) $(n,0)\in i$, $(n^{\prime},0)\in i^{\prime}$, and $n>n^{\prime}$, or 3) $(0,n)\in i$, $(0,n^{\prime})\in i^{\prime}$, and $n^{\prime}>n$. 

\subsection{Rational Numbers}
In mathematics, the \textbf{rational numbers} are ratios of integers. To define the rational numbers as sets, we first define a multiplication function $h$ on pairs of integers as follows:
\begin{equation}
    h:(a,b)\mapsto a\cdot b = \begin{cases}
        0 & \text{if} \ b=0\\
        (a\cdot c) + a & \text{if} \ b=c+1\\
        \textsf{sup}(\{a\cdot c:c<b\}) & \text{otherwise}\\
    \end{cases}
\end{equation}
To illustrate, consider $h(3,2)$. Since $2=1+1$, $h(3,2)=h(3,1)+3$. So, now we need to calculate $h(3,1)$. Since $1=0+1$, $h(3+1)=h(3,0)+3$. So, now we need to calculate $h(3,0)$, which is $0$. Putting this all together, we get:
$$h(3,2)=h(3,1)+3=(h(3,0)+3)+3=(0+3)+3=6.$$
Each rational number $q$ can be expressed as a ratio of integers $a$ and $b$. We denote such a ratio using the notation $[a/b]$, and define the ratio as the set:
$$q = [a/b] = \{(c,d)\in\mathbbm{Z}\times(\mathbbm{Z}\cup\{0\}):a\cdot d= b\cdot c\}.$$
This is the same thing as saying that a rational number $q$ that can be expressed as the ratio of integers $a$ and $b$ is the set of all pairs of integers $(c,d)$ such that $\frac{a}{b}=\frac{c}{d} = q$. Note that the second term in the pair cannot be zero or else the relation would be undefined. A rational number $q$ is positive if for all pairs of integers $(a,b)\in q$, $a$ and $b$ are both positive or both negative. Otherwise, $q$ is negative.

\subsection{Real Numbers}
Intuitively, the real numbers are the numbers used to measure \textit{continuous} quantities. Getting slightly more technical, we might say that the real numbers are used to measure quantities that can differ by an arbitrarily small amount. To get fully technical, we follow Richard Dedekind (1831-1916) and introduce the notion of a \textbf{Dedekind cut}. Let $\mathbbm{Q}$ be the set of all rational numbers. Let a rational number be positive if and only if for any pair of integers $(a,b)$ in the set that defines that rational number, either $a$ and $b$ are both positive or $a$ and $b$ are both negative. Define a relation $<$ on the rationals such that $<$ if and only if  $a\neq b$ and there exists a positive rational number $t$ such that $s = r + t$. A Dedekind cut is a set $A\subset\mathbbm{Q}$ such that:
\begin{enumerate}
    \item $A\neq \emptyset$.

    \item $A\neq \mathbbm{Q}$.

    \item If $x\in\mathbbm{Q}$, $y\in\mathbbm{Q}$, $x<y$, and $y\in A$, then $x\in A$ (i.e., $A$ extends infinitely into the negative rationals).

    \item If $x\in A$, then there exists a $y$ such that $x<y$ (i.e., $A$ has no greatest element).
\end{enumerate}
A Dedekind cut is so-named because it \textit{cuts} the rationals in half. For any Dedekind cut $A$, let $b$ be the unique rational number such that, for all rational numbers $c$, if $c\in A$ if and only if $c<b$. This is the \textbf{least upper bound} on $A$. Now consider the complement of $A$ in $\mathbbm{Q}$, $B=\mathbbm{Q}\setminus A$. One and only one of the following must be true of $B$:\ $b\in B$ or $b\not\in B$. If $b\in B$, then we identify the Dedekind cut $A$ with the real (and rational) number $b$. If, on the other hand, $b\not\in B$, then Dedekind proved that there is a unique, irrational, real number that ``fills the gap'' between $A$ and $B$. Thus, every real number can be defined set-theoretically as a specific Dedekind cut.\par

To illustrate, consider the real (and rational) number 2. We can define 2 as the Dedekind cut $A=\{a\in\mathbbm{Q}:a<2\}$. The real number 2 is the least upper bound of $A$, and is an element of $B=\mathbbm{Q}\setminus A$. However, now consider the irrational real number $\sqrt{2}$. We define $\sqrt{2}$ as the Dedekind cut $A=\{a\in\mathbbm{Q}:a^{2}<2\}$. Clearly, $\sqrt{2}$ is the least upper bound of $A$. Now consider $B=(\mathbbm{Q}\setminus A)=\{a\in\mathbbm{Q}:a^{2}\geq2\}$. The real number $\sqrt{2}$ is also not a member of this $B$, since it is not a rational number. However, we say that $\sqrt{2}$ just \textit{is} the space between $A$ and $B$. Thus, we identify $\sqrt{2}$ with the Dedekind cut $\{a\in\mathbbm{Q}:a^{2}<2\}$.


\subsection{Cardinal Numbers}
While ordinal numbers are used to rank the positions of objects, \textbf{cardinal numbers} are used to describe the \textit{size} of objects. In set theory, the \textbf{cardinality} of a set $X$ is written $|X|$. The cardinality of a set $X$ is determined by the least ordinal such that we can define a bijection between $X$ and that ordinal. For finite sets, this is straightforward. Consider the set $X=\{a,b,c,d\}$. The cardinality of the set is $4$ because the smallest (indeed, the only) ordinal such that we can define a bijection between $X$ and that ordinal is $$4=\{\emptyset,\{\emptyset\},\{\emptyset,\{\emptyset\}\},\{\emptyset,\{\emptyset\},\{\emptyset,\{\emptyset\}\}\}\}.$$
For infinite sets, the situation is somewhat more complicated. Consider the set of natural numbers $\mathbbm{N}$. Since the ordinal $\omega$ is the set of all natural numbers, we can define a bijection between $\mathbbm{N}$ and $\omega$. It turns out that there are many other ordinals $\omega^{\prime}$ such that $\omega<\omega^{\prime}$ for which we can also define a bijection between $\mathbbm{N}$ and $\omega^{\prime}$. But for any ordinal $o<\omega$, we cannot define a bijection between $\mathbbm{N}$ and $\omega$. We say that any set $X$ such that $\omega$ is the least ordinal such that a bijection can be defined between $X$ and $\omega$ has cardinality $\aleph_{0}$. The natural numbers $\mathbbm{N}$ have this cardinality, but so too do the integers $\mathbbm{Z}$ and the rationals $\mathbbm{Q}$. Any set with cardinality $\aleph_{0}$ is said to be \textbf{countably infinite}. To see why, note that for any of these sets, if we had infinite time, we could keep counting them without ever missing a number.\par 

One might think that the infinite cardinality $\aleph_{0}$ would be the largest cardinality; after all, what could be greater than infinity? This thought is wrong. Indeed, $\aleph_{0}$ is the \textit{smallest} of the infinite cardinalities. Let $\omega_{1}$ be the set of all finite or countably infinite ordinals (i.e., all ordinals that either have finite cardinality or have the same cardinality as $\omega$). By definition, $\omega_{1}$ is greater than any finite or countably infinite ordinal, since any finite or countably infinite ordinal is an element of $\omega_{1}$. We also know that for any ordinal $\omega^{\dagger}\neq\omega_{1}$ whose cardinality is neither finite nor countably infinite, $\omega^{\dagger}\not\in\omega_{1}$, since $\omega_{1}$ contains, by definition, all and only the ordinals of finite or countably infinite cardinality. Thus, $\omega_{1}$ is the least ordinal whose cardinality is strictly greater than any finite or countably infinite set, and so we call $\omega_{1}$ the least \textit{un}countable ordinal. Any set for which we can define a bijection into $\omega_{1}$ is said to have the cardinality $\aleph_{1}$, which is the smallest uncountable cardinality. The only smaller infinite cardinality than $\aleph_{1}$ is $\aleph_{0}$.\par

In 1891, Cantor famously proved a theorem that established that for any $X$, its power set $\mathcal{P}(X)$ is such that $|\mathcal{P}(X)|>|X|$. This turns out to establish an important fact about the real numbers:\ that they have uncountable cardinality. First, it is well-known that the set of real numbers $\mathbbm{R}$ has the same cardinality as the power set of the integers $\mathcal{P}(\mathbbm{Z})$. This means that the real numbers have a cardinality that is greater than $\aleph_{0}$. This raises an intriguing question:\ is the cardinality of the real numbers $\aleph_{1}$? The remarkable answer is that \textit{no one knows}. The hypothesis that the cardinality of the real numbers is $\aleph_{1}$ is known as the \textbf{continuum hypothesis}. Kurt G\"odel (1906-1978) proved that the ZFC axioms cannot disprove the continuum hypothesis, and Paul Cohen (1938-2007) proved that the ZFC axioms cannot prove the continuum hypothesis. As such, any argument that the continuum hypothesis is true or false will have to be, at least in part, a \textit{philosophical} argument that the ZFC axioms do not adequately capture the pre-theoretical concept of a set.\par 

\section{Conclusion}
It should be clear at this point that set theory is an \textit{extremely} rich subject, and indeed we have only begun to scratch the surface in this lesson. As mentioned above, the set-theoretic concepts established in this lesson will be used again in every subsequent topic that we cover.

\section*{Exercises}
\begin{enumerate}
    \item Prove that $\emptyset$ is a subset of all subsets of any set $X$.

    \item Prove that if $X\times Y = Y\times X$, then $X=Y$.

    \item Use the definition of a tuple as a set to prove that for any two tuples $(a,b,c)$ and $(c,b,a)$, $(a,b,c)$ = $(c,b,a)$ if and only if $a=c$.

    \item Prove that if $f:X\rightarrow Y$ is bijective, then for all $y\in Y$, $|f^{-1}(y)|=1$. 

    \item Use the ZFC axioms to prove that there is only one empty set.

    \item Use the ZFC axioms to prove that every set has a unique power set.

    \item Prove that the existence of the Russell set is inconsistent with the axioms of ZFC. (\textit{Hint:\ It helps to first prove that there is no set of all sets, using the power set and regularity axioms. Then, show how the Russell set can only be formed via the separation axiom if there is a set of all sets.}).

    \item Prove that $\omega\in(\omega+1)$ but $(\omega + 1)\not\in\omega$. 

    \item Let the function $\textsf{fusc}:\mathbbm{Z}^{+}\rightarrow\mathbbm{Z}^{+}$, where $\mathbbm{Z}^{+}$ is the set of positive integers, be defined as follows:
    \begin{equation*}
    \textsf{fusc}:z\mapsto \textsf{fusc}(z) = \begin{cases}
        0 & \text{if} \ z=0\\
        1 & \text{if} \ z=1\\
        \textsf{fusc}(2\cdot k) & \text{if} \ z\geq 1 \wedge z=2\cdot k\\
        \textsf{fusc}((2\cdot k)+1) & \text{if} \ z\geq 1 \wedge z=\textsf{fusc}(k) + \textsf{fusc}(k+1)\\
    \end{cases}
    \end{equation*}
        Let the \textbf{lowest terms pair} $(a,b)$ in any positive rational $q$ be the pair such that for any $(c,d)\in q$, $(c,d)=(\gamma \cdot a,\gamma \cdot b)$ where $\gamma$ is an integer greater than zero. It turns out that for any positive rational $q$, its lowest terms pair $(a,b)$ is such that $(a,b)=(\textsf{fusc}(z),\textsf{fusc}(z+1))$ for some $z$. Use this fact to prove that that rational numbers $\mathbbm{Q}$ have cardinality $\aleph_{0}$ (\textit{Hint:\ you'll have to find a way to deal with negative rationals!})
\end{enumerate}

\end{document}
